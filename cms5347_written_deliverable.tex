\documentclass[notitlepage,12pt]{article}
\author{Christopher Sasarak}
\date{2/9/2013}
\title{Reliability of the Kademlia DHT}
\begin{document}
\maketitle

\section{Overview}
\label{sec:over}

This project is inspired by the Tor and Invisible Internet Project (I2P) anonymity protocols. These
protocls seek to make internet usage anonymous by making encrypted
networks for public use. Each of them does this a different way. Tor
is a circuit switched network which makes use of onion
routing\cite{tor}. I2P in contrast, is a message based system which
makes use of garlic routing, a modified form of onion routing
\cite{i2p}. In this project, the Kademlia algorithm, a modified form
of which is used in the routing of I2P, is examined to see how
reliable it is when nodes are leaving the system. I hypothesize that
as the rate of nodes leaving the system increases, the amount of time
that it takes for a given node to perform a look-up will increase as
well.

\section{Papers}
\label{sec:papers}

\subsection{Paper 1: Performance analysis of anonymous communication channels provided by Tor}
\label{sec:Tor}

In \emph{Performance analysis of anonymous communication channels
  provided by Tor} the authors perform an experimental analysis of Tor
in order to find out how it performs and what might be an obstacle for
users trying to use the system \cite[p.1]{tor}. 

Tor stands for \emph{The Onion Router} and is a system which provides
anonymous internet access to users. It does this by randomly
generating paths for packets and utilizing a technology called
\emph{onion routing}. Onion routing is a method of communication where
a message is encrypted for each hop on the network, and as it
traverses each hop one layer of encryption is removed \cite[p. 1]{tor}

Unfortunately, Tor is quite slow. One factor is that Tor is a network
of comprised of volunteers who create nodes (\emph{Onion Routers})
along which messages can travel\cite[p. 2]{tor}. At the time that
\cite{tor} was written the number of permanent onion routers was about
one thousand while there were several hundred thousand users. This
disparity between routers and users leads to significant slow-downs
which discourage users from using the network.  The authors want to
investigate why ways in which Tor's performance can be improved
because the distributed nature of Tor makes the network suffer when
large numbers of users leave \cite[p. 1]{tor}.

\bibliographystyle{plain}
\bibliography{citations}
\end{document}


