\documentclass[notitlepage,12pt]{article}
\author{Christopher Sasarak}
\date{2/9/2013}
\title{Reliability of the Kademlia DHT}
\begin{document}
\maketitle

\section{Overview}
\label{sec:over}

This project is inspired by the Tor and Invisible Internet Project (I2P) anonymity protocols. These
protocls seek to make internet usage anonymous by making encrypted
networks for public use. Each of them does this a different way. Tor
is a circuit switched network which makes use of onion
routing\cite{tor}. I2P in contrast, is a message based system which
makes use of garlic routing, a modified form of onion routing
\cite{i2p}. In this project, the Kademlia algorithm, a modified form
of which is used in the routing of I2P, is examined to see how
reliable it is when nodes are leaving the system. I hypothesize that
as the rate of nodes leaving the system increases, the amount of time
that it takes for a given node to perform a look-up will increase as
well.

\section{Papers}
\label{sec:papers}

\subsection{Paper 1: Performance analysis of anonymous communication channels provided by Tor}
\label{sec:Tor}

In \emph{Performance analysis of anonymous communication channels
  provided by Tor} the authors perform an experimental analysis of Tor
in order to find out how it performs and what might be an obstacle for
users trying to use the system \cite[p.1]{tor}

\bibliographystyle{plain}
\bibliography{citations}
\end{document}


